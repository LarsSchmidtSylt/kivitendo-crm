%%%%%%%%%%%%%%%%%%%%%%%%%%%%%%%%%%%%%%%%%%
% Diese Vorlage steht unter der GPL-Lizenz, Version2 
% siehe  http://www.gnu.de/gpl-ger.html
%%%%%%%%%%%%%%%%%%%%%%%%%%%%%%%%%%%%%%%%%%
\documentclass[a4paper,foldmarks=true]{scrartcl}
\usepackage{fancyhdr}
\usepackage[pdftex]{graphicx}
\usepackage{german}
\usepackage{icomma}
\usepackage{longtable}
\usepackage[utf8]{inputenc}
\usepackage{pifont}
\usepackage{setspace}
\usepackage{textcomp}
\usepackage{ifthen}
\usepackage{filecontents}
\usepackage{ltxtable}
\usepackage{booktabs}
\usepackage{etex}
\usepackage{lastpage}
\usepackage{multirow} 
\usepackage{color}
\setlength{\voffset}{-0.8cm} 
\setlength{\hoffset}{-1.5cm}
\setlength{\topmargin}{0cm}
\setlength{\headheight}{0.5cm}
\setlength{\headsep}{0.1cm}
\setlength{\topskip}{0pt}
\setlength{\oddsidemargin}{1.0cm}
\setlength{\evensidemargin}{1.0cm}
\setlength{\textwidth}{18.1cm}
\setlength{\textheight}{24.3cm}
\setlength{\footskip}{1.8cm} 
\setlength{\parindent}{0pt}
\renewcommand{\baselinestretch}{1}

%\input{fancy.tex}
%\input{deutsch.tex}

% ------------------------------------------------------------
% Der eigtl. Dokumenten-Inhalt beginnt hier:
% ------------------------------------------------------------
% Font für Code 39
\font\xlix=wlc39 scaled 1200
\newcommand{\barcode}[1]{{\xlix@#1@}}
\newcommand\WG[1]{%
    \\
    & \multicolumn{2}{c}{\textbf{#1}} & \\
    \hline
}
\newcommand\Zeile[4]{%
                 #2 &
                 \multicolumn{1}{r}{\small #3} &
                 \multirow{2}{50px}{\includegraphics*[height=55px]{#1}} \\
                 \multicolumn{2}{X}{\raggedright\small #4} & \\ 
}


\begin{document}
% Schrift-Stil festlegen:
\fontfamily{cmss}\fontsize{12}{15pt plus 0.15pt minus 0.1pt}\selectfont

% Erste Seite
%\thispagestyle{briefkopf}

% ------------------------------------------------------------
% Dokument-Info, Bemerkungen, Datum, Bearbeiter, ...
% ------------------------------------------------------------

%\vspace{4mm}

% ------------------------------------------------------------
% Die Positionen-Tabelle
% ------------------------------------------------------------
\begin{filecontents}{tabelle.tex}

%\begin{longtable}{@{}p{100px}lllp{100px}@{ }@{}} % Definition der Spalten
%\begin{longtable}{@{}lXll@{ }@{}} % Definition der Spalten
\begin{longtable}{@{}XXp{100px}@{}} % Definition der Spalten
	% Spaltenueberschriften

	\hline % Trennlinie
    \endfirsthead

	\hline % Trennlinie
    \endhead

    % Fuss der Teiltabellen
    %\multicolumn{3}{r}{Fortsetzung n\"achste Seite. Seite: \thepage{} von \pageref{LastPage}} \\
  \endfoot

  % Das Ende der Tabelle
	\hline % Trennlinie
  \endlastfoot

	% Artikel-Positionen:
	<%foreach_part%>   
            \ifthenelse{\equal{<%newpg%>}{new}}{ \WG{<%partsgroup%>} } 
            { \Zeile{<%image%>}{<%partnumber%>}{<%sellprice%>}{<%description%>} }
	<%end_part%>

	\hline
\end{longtable}
\end{filecontents}
\LTXtable{\textwidth}{tabelle.tex}

\vspace{5mm}

\label{LastPage}
\end{document}
